\documentclass[12pt, letterpaper]{article}
\usepackage{amsmath} % Required for mathematical symbols
\usepackage{amssymb} % Required for more mathematical symbols
\usepackage{amsthm}
\usepackage{graphicx} % Required for inserting images
\usepackage{enumitem} % For smaller bullet points
\usepackage{hyperref}
\usepackage{geometry}
 \geometry{
 a4paper,
 total={170mm,257mm},
 left=20mm,
 top=20mm,
 }

\title{Informatik 1 Übung 1}
\author{Lars Schlichting, Hendrik Kleine Vennekate}
\date{}

\begin{document}

\maketitle

\paragraph{Aufgabe 1}

In der Vorlesung haben Sie die Konkatenation von Wörtern kennengelernt. Betrachten Sie die Wörter $u = Hallo, v = Welt, w = \epsilon$. Geben Sie die durch Konkatenation gebildeten Wörter $uv, wv, uwwu$ an. Welche Länge besitzen diese Wörter?

\begin{enumerate}
    \item $uv$: $HalloWelt$, $\mid uv \mid=9$ 
    \item $wv$: $Welt$, $\mid wv \mid=4$ 
    \item $uwwu$: $HalloHallo$, $\mid uv \mid=10$ 
\end{enumerate}

\paragraph{Aufgabe 2}

Wie Sie wissen, bezeichnet $\sum^{*}$ die Menge aller Wörter über dem Alphabet $\sum$.

\paragraph{(a)}
Die Menge der Wörter über dem leeren Alphabet beträgt \textbf{1}, da das leere Wort $\epsilon$ ein Wort über $\sum$ ist. 

\paragraph{(b)}
(i)\\

\noindent Alle Präfixe des Wortes $100$:
\begin{enumerate}
    \item $1$, $r = 00$
    \item $10$, $r = 0$
    \item $100$, $r = \epsilon$
\end{enumerate}
\noindent Alle Suffixe des Wortes $100$:
\begin{enumerate}
    \item $0$, $t = 10$
    \item $00$, $t = 1$
    \item $100$, $t = \epsilon$
\end{enumerate}

\noindent (ii) Da das Wort \(100\) nur die Zeichen \(1\) und \(0\) enthält, können wir daraus schließen, dass das zugrunde liegende Alphabet \(\Sigma = \{0, 1\}\) = binäres Alphabet ist.


\paragraph{Aufgabe 3}

Gegeben sei die Turingmaschine $M = (Q, \Sigma, \Gamma, \delta, q_1, \# , \{q_6\})$ mit
\begin{itemize}
    \item $Q = \{q_1, \ldots, q_6\}$,
    \item $\Sigma = \{0, 1, .\}$ (drei Zeichen, nämlich Null, Eins und Punkt),
    \item $\Gamma = \Sigma \cup \{\#\}$ und
    \item $\delta$ definiert durch folgende Tabelle:
\end{itemize}

\[
\begin{array}{|c|c|c|c|c|}
\hline
   & 0 & 1 & . & \# \\
  \hline
  q_1 & q_2, 0, R & q_2, 1, R & q_5, ., S & q_5, \#, S \\
  q_2 & q_2, 0, R & q_2, 1, R & q_3, ., R & q_5, \#, S \\
  q_3 & q_4, 0, R & q_4, 1, R & q_5, ., S & q_5, \#, S \\
  q_4 & q_4, 0, R & q_4, 1, R & q_5, ., S & q_6, \#, S \\
  q_5 & q_5, 0, S & q_5, 1, S & q_5, ., S & q_5, \#, S \\
  \hline
\end{array}
\]\\


\noindent Von $M$ werden alle Wörter akzeptiert, in denen mind. ein "." vorkommt. Vor und nach dem Punkt dürfen beliebig viele $0$, oder $1$ stehen, jedoch muss mind. 1 vorhanden sein. Am Ende des Wortes muss eine Raute stehen, um die Maschine in den Endzustand $q_6$ zu überführen. Alles andere ist unzulässig und führt nie zur Terminierung der Maschine. Alle Zeichen werden während dieses Vorgangs nicht verändert.

\begin{enumerate}
    \item Zustand prüft, ob das Wort mit einer Binärzahl startet
    \item Zustand sorgt dafür, dass man beliebig viele weitere Binärzahlen hinzufügen kann und überführt in Zustand 3 im Falle eines Punktes.
    \item Zustand prüft analog zu Zustand 1, dass dem Punkt eine Binärzahl folgt.
    \item Zustand erlaubt, dass beliebig viele Binärzahlen bis zum Ende des Wortes folgen dürfen. Überführt beim Wortende in Endzustand.
    \item Zustand sorgt für einen Stop des LS-Kopfes.
\end{enumerate}

\paragraph{Aufgabe 4}

\noindent Als zusätzliches Zeichen haben wir das $a$ genommen, um als Markierung für das vorab Gezählte zu dienen. Dieses taucht maximal 1 mal zeitgleich auf.

Gegeben sei die Turingmaschine $M = (Q, \Sigma, \Gamma, \delta, q_1, \# , \{q_8\})$ mit
\begin{itemize}
    \item $Q = \{q_1, \ldots, q_8\}$,
    \item $\Sigma = \{0, 1\}$,
    \item $\Gamma = \Sigma \cup \{\#\}$ und
    \item $\delta$ definiert durch folgende Tabelle:
\end{itemize}

\[\]
\begin{array}{|c|c|c|c|c|}
\hline
\text{} & 0 & 1 & a & \# \\
\hline
q_1 & q_2, a, L & q_2, a, L & q_2, a, S & q_8, 0, S \\
q_2 & q_3, 1, R & q_2, 0, L & q_2, a, S & q_3, 1, R \\
q_3 & q_3, 0, R & q_3, 1, R & q_4, \#, R & q_3, \#, R \\
q_4 & q_5, a, L & q_5, a, L & q_2, a, S & q_6, \#, L \\
q_5 & q_3, 1, R & q_2, 0, L & q_2, a, S & q_5, \#, L \\
q_6 & q_7, 0, L & q_7, 1, L & q_2, a, S & q_6, \#, L \\
q_7 & q_7, 0, L & q_7, 1, L & q_2, a, S & q_8, \#, R \\
\hline
\end{array}
\]

\begin{enumerate}
    \item Zustand detektiert das erste Bit und überschreibt dieses mit $a$. Falls das Eingabewort leer sein sollte, wird eine 0 geschrieben und in den Endzustand 8 gewechselt.
    \item Zustand addiert binär eine 1 auf unsere Summe und geht anschließend in Zustand 3.
    \item Zustand geht solange nach rechts, bis er auf die Markierung $a$ trifft, diese entfernt und in Zustand 4 wechselt.
    \item Zustand schreibt bei einer $1$, oder $0$ ein $a$, um die neue Markierung zu setzen und in Zustand 5 zu wechseln. Falls das Wort endet, wird in Zustand 6 gegangen.
    \item Zustand geht solange nach links, bis er auf die Summe trifft. Wenn das LSB eine 0 ist, wird eine 1 dazuaddiert und wird wieder iterativ in Zustand 3 gegangen. Bei einer 1 haben wir einen Übertrag und dieser wird in Zustand 2 behandelt.
    \item Zustand sorgt dafür, dass wir zum Ende unserer Summe kommen und überführt in Zustand 7.
    \item Zustand sorgt dafür, dass der Lese-/Schreibkopf auf dem MSB endet und in den Endzustand 8 überführt.
\end{enumerate}

\noindent Im Abgabeordner finden Sie außerdem eine JSON-Datei, die in den auf dem Übungsblatt stehenden Turing Machine Simulator importiert werden kann. Auf der Website muss man jedoch den Leseschreibkopf auf das erste Bit ziehen und den Startzustand auf $1$ setzen.

\end{document}
